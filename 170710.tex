%%%%%%%%%%%%%%%%%%%%%%%%%%%%%%%%%%%%%%%%%
% Medium Length Professional CV
% LaTeX Template
% Version 2.0 (8/5/13)
%
% This template has been downloaded from:
% http://www.LaTeXTemplates.com
%
% Original author:
% Trey Hunner (http://www.treyhunner.com/)
%
% Important note
% This template requires the resume.cls file to be in the same directory as the
% .tex file. The resume.cls file provides the resume style used for structuring the
% document.
%
%%%%%%%%%%%%%%%%%%%%%%%%%%%%%%%%%%%%%%%%%

%----------------------------------------------------------------------------------------
%	PACKAGES AND OTHER DOCUMENT CONFIGURATIONS
%----------------------------------------------------------------------------------------

\documentclass{resume} % Use the custom resume.cls style

\usepackage[left=0.75in,top=0.6in,right=0.75in,bottom=0.6in]{geometry} % Document margins

\name{Kyle W. Hershey} % Your name
\address{2296 Long Avenue \\ Saint Paul, MN 55114} % Your address
\address{(608)~$\cdot$~345~$\cdot$~8595 \\ kwhershey@gmail.com} % Your phone number and email

\begin{document}
%----------------------------------------------------------------------------------------
%	SUMMARY SECTION
%----------------------------------------------------------------------------------------

\begin{rSection}{Summary of Qualifications}
\item Ph.D in Materials Science \& Engineering (Expected Summer 2018)
%\item Specific expertise in characterization of kinetic processes of electronic species within OLEDs in the transient, steady-state and operational lifetime regimes
\item Primary hardware and software designer for multiple permanent laboratory measurement and sensor apparatus
\item Experience implementing software control of testing equipment along with database design for data management 
\item Data analysis and data mining to enhance experimental techniques
\item Extensive experience in the fabrication and characterization of thin films via solution and vapor deposition methods
\item Testing and analysis of Organic Light-Emitting Devices (OLEDs), along with failure analysis
\item Strong leader with experience in mentoring and financial management 


\end{rSection}

%----------------------------------------------------------------------------------------
%	EDUCATION SECTION
%----------------------------------------------------------------------------------------

\begin{rSection}{Education}
\hspace{-.5em}{\bf University of Minnesota} \hfill { Expected Summer 2018} \\
Ph.D. in Materials Science \& Engineering, Holmes Research Group \hfill GPA: 3.545\\
{\em Transient and Operational Lifetime Dynamics of Organic Light Emitting Devices (OLEDs)}\vspace{-1ex}

\hspace{-.5em}{\bf Coe College} \hfill { June 2013} \\ 
B.A.s in Physics, Mathematics, and Computer Science \hfill GPA: 3.927
\end{rSection}

%----------------------------------------------------------------------------------------
%	WORK EXPERIENCE SECTION
%----------------------------------------------------------------------------------------

\begin{rSection}{Research Experience}

\begin{rSubsection}{University of Minnesota, Chemical Engineering and Materials Science}{July 2013 - Present}{Graduate Student, Prof. Russell J Holmes}{Minneapolis, MN}
\vspace{-.5 em}
\hspace{-1em}{\em Supported via UMN-Dow Chemical University Partnership Initiative} \smallskip \\
%{\em Supported via UMN-Dow Chemical University Partnership Initiative} \smallskip \\
\begin{rSubsubsection}{Experimental}
\item Actively collaborated with the Dow Chemical Company as an industrial sponsor in order to provide methods of understanding device and chemical degradation of materials 
\item Theoretical and experimental development for novel OLED lifetime testing method 
\item Enhanced understanding of degradation pathways of OLEDs
\item Developed methods of enhancing device efficiency through an improved understanding of dynamic processes 
\item Model development for fitting of experimental data of transient and steady-state electroluminescence 
\item Fabrication and characterization of OLEDs at transient, steady-state, and lifetime timescales
\end{rSubsubsection}
\begin{rSubsubsection}{Data Science}
\item Created multiple computer controlled data acquisition systems, including hardware and software engineering
\item Automated collection from experimental measurement and sensor equipment into database for analysis
%\item Integrated data analysis into acquisition software and programmed central database system for all users
\item Utilized data mining and machine learning techniques to analyze experiemntal results
\item Wrote Python software to enable rapid experimental comparison
\item Developed sophisticated graphical user interfaces in Python for data collection and analysis
\item Collaborated with departmental IT to create a research archive, fully tagged with experimental metadata
\item Created permanent hardware setups for numerous experimental apparatus
\item Designed and multiplexed OLED lifetime measurement setup to 14 simultaneous devices
%\item Database setup for test data storage and analysis
%\item Maintain code repository for testing and analysis software using Github, featuring 13,000+ lines of code
\item Extensive use of software connection to testing equipment including electrical supplies, measurement units, spectrometers, and custom microprocessor board based electronics
\end{rSubsubsection}
\end{rSubsection}

%-----------------------------------------------

\begin{samepage}
\begin{rSubsection}{Northwestern University, Materials Science/Chemistry }{June 2012 - August 2012}{REU Student, Prof. George Schatz, Chemistry}{Evanston, IL}

\item {\em Field Enhancement Due to Plasmonic Nanostructures}
\item Finite element modeling of electromagnetic field enhancement around gold dimers
\item Cluster computing on Northwestern's high performance computing system - {\em Quest}
\end{rSubsection}
\end{samepage}

%------------------------------------------------

\begin{rSubsection}{Rockwell Collins, Inc., Advanced Technology Center}{June 2011 - August 2011}{Summer Intern}{Cedar Rapids, IA}
%Microelectronics Packaging Unit
\item {\em Microelectronics die attach process development}
\item Developed methodology for attaching microelectronics using various techniques for low stand-off applications
\item Composed internal documentation on microelectronics die attachment and Transient Liquid Phase bonding
\end{rSubsection}

%------------------------------------------------

\begin{samepage}
\begin{rSubsection}{Coe College}{June 2010 - August 2010}{REU Student, Prof. James Cottingham, Coe College, Physics}{Cedar Rapids, IA}
\item Examined the effects of the vibrations in the pipe walls of free-reed wind instruments
\item Materials measurements of density and Young's modulus of bamboo pipes and simulation of vibrational modes
\end{rSubsection}
\end{samepage}

\end{rSection}

%----------------------------------------------------------------------------------------
%	TECHNICAL STRENGTHS SECTION
%----------------------------------------------------------------------------------------
\begin{rSection}{Technical Strengths}

\begin{tabular}{ @{} >{\hspace{-0.5em}\bfseries}l @{\hspace{3ex}} p{14.25cm} }
Techniques & Ellipsometry, Electronic Device Characterization, UV/Visible Spectrometry, Transient fluorescence measurements (electrically and optically pumped), electrical and optical degradation, OLED lifetime characterization, Scanning Electron Microscopy (SEM), Optical Microscopy, Optical Field Modeling, Transfer Matrix Formalism, Finite Difference Time Domain (FDTD) Modeling, Transient Liquid Phase (TLP) bonding, AuSn eutectic bonding, solder bumping, Ball Grid Array (BGA) attachment, cross sectional die analysis \\
Equipment & Thermal evaporation vacuum chamber, glovebox, spin coater, UV ozone, sonicator, pulse generator, impedance spectrometer, frequency generator, optical microscope, SEM, ellipsometer, oscilloscope, FFT audio spectrometer, probe station, cryogenic probe station, source meter, spectrometer, Arduino, Pulsed and CW lasers, Class 10000 clean room, chip bonder, wire bonding, stud bumping, plating baths \\
Software & Matlab, Mathematica, Anaconda, Autodesk Inventor, AutoCAD, Solidworks, OriginLabs, ChemDraw, Powershell, Microsoft Office Suite, Github, KiCAD, \LaTeX , Linux(Ubuntu, Debian, Red Hat, Arch) Vim, SSH, SCP, VNC, Bash \\
Programming & Python, Pandas, Numpy, Scipy, Plotly, Matplotlib, C, C++, C\#, Objective C, Matlab, Tk graphics, National Instrument VISA command library, HTML, PHP, MongoDB, SQLite, SQL  \\
\end{tabular}

\end{rSection}

%----------------------------------------------------------------------------------------
% LEADERSHIP 
%----------------------------------------------------------------------------------------

\begin{rSection}{Leadership Experience}

\begin{rSubsection}{Holmes Group Purchasing Officer}{2016 - Present}{University of Minnesota}{}
\item Served as group purchasing officer, responsible for handling all purchases less than \$500
\item Regular interaction with the Accounting Office to process transactions
\item Weekly transaction accounting and justification
\end{rSubsection}

\begin{rSubsection}{Mentor of Undergraduate and High School Research Students}{2016 - Present}{University of Minnesota}{}
\item Primary contact for undergraduate student developing new technique for solution lifetime measurements
\item Oversaw two high school students in creating organic lasers for a summer research experience project
\item Taught advanced research topics at basic level to enable understanding of lab work
\end{rSubsection}

\begin{rSubsection}{Executive Board Member}{2011 - 2013}{\textbf{Physics Club:} Treasurer (2011), Vice President (2012)}{\textbf{Soc. of Physics Students:} President (2012 - 2013)}
\item Assisted in organization of {\em Playground of Science}, an annual outreach event for 1,000+ elementary school students
\item Defended Physics Club at annual student activities committee budget meeting and special budgetary meetings which succeeded in funding 30+ students to attend the 2012 Sigma Pi Sigma Quadrennial Congress
\item Represented the Society of Physics Students chapter at the 2012 Sigma Pi Sigma Quadrennial Congress
\end{rSubsection}

\begin{rSubsection}{Senior Patrol Leader, Boy Scouts of America}{2006}{Troop 46 Glendale, AZ}{}
\item Organized and led weekly troop meetings and monthly campouts for a troop of 30+ boys aged 12-16
\end{rSubsection}

\end{rSection}

%----------------------------------------------------------------------------------------
% TEACHING 
%----------------------------------------------------------------------------------------

\begin{rSection}{Teaching Experience}
\begin{rSubsection}{Teaching Assistant, Senior Design (MATS 4400)}{2017}{University of Minnesota}{}
\item Mentored 30+ students working in groups with industrial partners designing solutions to commercially relevant problems
\item Assisted in idea generation, design specification, technical calculations and financial analysis
\item Evaluated biweekly presentations and written reports of projects
\end{rSubsection}
\pagebreak

\begin{rSubsection}{Teaching Assistant, Materials Performance (MATS 4221)}{2015}{University of Minnesota}{}
\item Ran two laboratory sections every week, seeing 30 students
\item Oversaw laboratory experiments for thermal and mechanical characterization, including thermal stress, creep, fatigue and stress-strain testing
\end{rSubsection}

\end{rSection}
%----------------------------------------------------------------------------------------
% HONORS	
%----------------------------------------------------------------------------------------

\begin{rSection}{Honors}

\item Member of the Materials Research Society (2016 - Present)
\item University of Minnesota College of Science and Engineering Fellowship (2013 - 2014)
\item Member of Phi Beta Kappa (Inducted 2013)
\item Member of Sigma Pi Sigma Physics Honors Society (Inducted 2013)
\item Dean's List, Coe College (2009 - 2013)
\item Eagle Scout (2008)

\end{rSection}

%----------------------------------------------------------------------------------------
%PUBLICATIONS  and PRESENTATIONS SECTION	
%----------------------------------------------------------------------------------------

\begin{rSection}{Publications and Presentations}
\begin{rSubsection}{Journal Publications}{}{}{}
\item J Bangsund, \textbf{KW Hershey}, RJ Holmes. {\em Origin of Lifetime Enhancement in Mixed Emissive Layer Organic Light-Emitting Devices} (Submitted to ACS Applied Materials and Interfaces).
\item  \textbf{KW Hershey}, J Suddard-Bangsund, G Qian, RJ Holmes. {\em Decoupling Degradation in Exciton Formation and Recombination During Lifetime Testing of Organic Light-Emitting Devices}. Applied Physics Letters. 2017.
\item F Xu, \textbf{KW Hershey}, RH Holmes, TR Hoye. {\em Blue-Emitting Arylalkynyl Naphthalene Derivatives via a Hexadehydro-Diels-Alder Cascade Reaction} . Journal of the American Chemical Society. 2016 138 (39), 12739-12742
\item \textbf{KW Hershey}, RJ Holmes. {\em Unified Analysis of Transient and Steady-State Electrophosphorescence Using Exciton and Polaron Dynamics Modeling} . Journal of Applied Physics. 2016, 120 (19), 195501
\item \textbf{KW Hershey}, JP Cottingham. {\em Matierial Properties of Pipes of Reeds From the Southeast Asian Khaen} . Journal of the Acoustics Society of America. 2011,129 (4) 2520

\end{rSubsection}

%----------------------------------------------------------------------------------------

\begin{rSubsection}{Oral Presentations}{}{}{}
\item \textbf{KW Hershey}, RJ Holmes.{\em Decoupling Exciton Formation and Recombination Losses in Organic Light-Emitting Devices During Lifetime Testing}, Optical Society of America - Light, Energy and the Environment Congress. Boulder, CO. November 2017
\item \textbf{KW Hershey}, RJ Holmes.{\em Decoupling Degradation Mechanisms During Lifetime Testing of Organic Light-Emitting Devices}, UMN IPrime. Minneapolis, MN. June 2017 
\item \textbf{KW Hershey}, RJ Holmes.{\em Modeling Exciton and Polaron Dynamics to Analyze OLED Behavior}, UMN IPrime. Minneapolis, MN. June 2016 
\item \textbf{KW Hershey}, RJ Holmes.{\em Modeling Exciton and Polaron Dynamics to Analyze OLED Behavior}, MRS Spring Conference. Phoenix, AZ. April 2016 
\item \textbf{KW Hershey}, JP Cottingham. {\em Material Properties of Pipes and Reeds from the Southeast Asian Khaen}, Acoustical Society of America National Meeting. Seattle, WA. May 2011 
\end{rSubsection}
\begin{rSubsection}{Poster Presentations}{}{}{}
\item \textbf{KW Hershey}, RJ Holmes.{\em Decoupling Degradation Mechanisms During Lifetime Testing of Organic Light-Emitting Devices}, UMN IPrime. Minneapolis, MN. June 2017 
\item \textbf{KW Hershey}, RJ Holmes.{\em Connecting Transient and Steady-State Dynamics in Organic Light Emitting Devices}, UMN IPrime. Minneapolis, MN. June 2016 
\item \textbf{KW Hershey}, RJ Holmes.{\em Transient Analysis of Organic Light-emitting Devices}, UMN IPrime. Minneapolis, MN. May 2015 
\item \textbf{KW Hershey}, JP Cottingham.{\em Material Properties of Pipes and Reeds from the Southeast Asian Khaen}, Sigma Pi Sigma Quadrennial Physics Congress. Orlando, FL. April 2012.

\end{rSubsection}
\end{rSection}

\end{document}
